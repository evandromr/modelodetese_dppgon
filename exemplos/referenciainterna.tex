% ---
\section{Remissões internas}
% ---

Ao nomear a \autoref{tab-nivinv} e a \autoref{fig_circulo}, apresentamos um
exemplo de remissão interna, que também pode ser feita quando indicamos o
\autoref{cap_exemplos}, que tem o nome \emph{\nameref{cap_exemplos}}. O número
do capítulo indicado é \ref{cap_exemplos}, que se inicia à
\autopageref{cap_exemplos}\footnote{O número da página de uma remissão pode ser
obtida também assim:
\pageref{cap_exemplos}.}.
Veja a \autoref{sec-divisoes} para outros exemplos de remissões internas entre
seções, subseções e subsubseções.

O código usado para produzir o texto desta seção é:

\begin{verbatim}
Ao nomear a \autoref{tab-nivinv} e a \autoref{fig_circulo}, apresentamos um
exemplo de remissão interna, que também pode ser feita quando indicamos o
\autoref{cap_exemplos}, que tem o nome \emph{\nameref{cap_exemplos}}. O número
do capítulo indicado é \ref{cap_exemplos}, que se inicia à
\autopageref{cap_exemplos}\footnote{O número da página de uma remissão pode ser
obtida também assim:
\pageref{cap_exemplos}.}.
Veja a \autoref{sec-divisoes} para outros exemplos de remissões internas entre
seções, subseções e subsubseções.
\end{verbatim}
% ---
