% ---
\section{Divisões do documento: seção}\label{sec-divisoes}
% ---

Esta seção testa o uso de divisões de documentos. Esta é a
\autoref{sec-divisoes}. Veja a \autoref{sec-divisoes-subsection}.

\subsection{Divisões do documento: subseção}\label{sec-divisoes-subsection}

Isto é uma subseção. Veja a \autoref{sec-divisoes-subsubsection}, que é uma
\texttt{subsubsection} do \LaTeX, mas é impressa chamada de ``subseção'' porque
no Português não temos a palavra ``subsubseção''.

\subsubsection{Divisões do documento: subsubseção}
\label{sec-divisoes-subsubsection}

Isto é uma subsubseção.

\subsubsection{Divisões do documento: subsubseção}

Isto é outra subsubseção.

\subsection{Divisões do documento: subseção}\label{sec-exemplo-subsec}

Isto é uma subseção.

\subsubsection{Divisões do documento: subsubseção}

Isto é mais uma subsubseção da \autoref{sec-exemplo-subsec}.


\subsubsubsection{Esta é uma subseção de quinto
nível}\label{sec-exemplo-subsubsubsection}

Esta é uma seção de quinto nível. Ela é produzida com o seguinte comando:

\begin{verbatim}
\subsubsubsection{Esta é uma subseção de quinto
nível}\label{sec-exemplo-subsubsubsection}
\end{verbatim}

\subsubsubsection{Esta é outra subseção de quinto nível}\label{sec-exemplo-subsubsubsection-outro}

Esta é outra seção de quinto nível.


\paragraph{Este é um parágrafo numerado}\label{sec-exemplo-paragrafo}

Este é um exemplo de parágrafo nomeado. Ele é produzida com o comando de
parágrafo:

\begin{verbatim}
\paragraph{Este é um parágrafo nomeado}\label{sec-exemplo-paragrafo}
\end{verbatim}

A numeração entre parágrafos numeradaos e subsubsubseções são contínuas.

\paragraph{Esta é outro parágrafo numerado}\label{sec-exemplo-paragrafo-outro}

Esta é outro parágrafo nomeado.

% ---
\section{Este é um exemplo de nome de seção longo. Ele deve estar
alinhado à esquerda e a segunda e demais linhas devem iniciar logo abaixo da
primeira palavra da primeira linha}
% ---

Isso atende à norma \citeonline[seções de 5.2.2 a 5.2.4]{NBR14724:2011} 
 e \citeonline[seções de 3.1 a 3.8]{NBR6024:2012}.
 % ---
