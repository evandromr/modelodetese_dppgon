% -----------------------------------------------------------------------------
% Pacotes fundamentais 
% -----------------------------------------------------------------------------
\usepackage{cmap}				% Mapear caracteres especiais no PDF
\usepackage[T1]{fontenc}		% Selecao de codigos de fonte.
\usepackage{lastpage}			% Usado pela Ficha catalográfica
\usepackage{indentfirst}		% Indenta o primeiro parágrafo de cada seção.
\usepackage{color}				% Controle das cores
\usepackage{graphicx}			% Inclusão de gráficos
\usepackage{hyphenat}           % pode forçar palavras a não serem hyphenizadas
% -----------------------------------------------------------------------------

% -----------------------------------------------------------------------------
% Fonte
% -----------------------------------------------------------------------------
% A fonte padrão do LaTeX é 'Computer Modern' a fonte 'Latin Modern' oferece
% algumas melhorias na renderização de símbolos

\usepackage{lmodern}	        % Usa a fonte Latin Modern
% -----------------------------------------------------------------------------
		
% -----------------------------------------------------------------------------
% Pacotes de citações
% -----------------------------------------------------------------------------
% o pacote abaixo permite citações do tipo: "citado na página X." etc.
\usepackage[brazilian,hyperpageref]{backref} % Paginas com as citações

% veja a documentação do pacote abntex2cite para instruções no padrão de
% citação (veja referência e licensa no arquivo principal desse projeto)
\usepackage[alf]{abntex2cite}	% Citações padrão ABNT do tipo: Autor, Ano
% -----------------------------------------------------------------------------

% -----------------------------------------------------------------------------
% Pacotes adicionais
% -----------------------------------------------------------------------------
\usepackage{lipsum}				% para geração de lero-lero em latim
% -----------------------------------------------------------------------------

% -----------------------------------------------------------------------------
% CONFIGURAÇÕES DE PACOTES
% -----------------------------------------------------------------------------
% Configurações do pacote backref para citações indiretas
% Usado sem a opção hyperpageref de backref
\renewcommand{\backrefpagesname}{Citado na(s) página(s):~}
% Texto padrão antes do número das páginas
\renewcommand{\backref}{}
% Define os textos da citação
\renewcommand*{\backrefalt}[4]{
	\ifcase #1 %
		Nenhuma citação no texto.%
	\or
		Citado na página #2.%
	\else
		Citado #1 vezes nas páginas #2.%
	\fi}%
% -----------------------------------------------------------------------------

% -----------------------------------------------------------------------------
% Espaçamentos entre linhas e parágrafos 
% -----------------------------------------------------------------------------
% O tamanho do parágrafo é dado por:
\setlength{\parindent}{1.3cm}
% Controle do espaçamento entre um parágrafo e outro:
\setlength{\parskip}{0.2cm}  % tente também \onelineskip
% -----------------------------------------------------------------------------

% -----------------------------------------------------------------------------
% informações do PDF
% -----------------------------------------------------------------------------
\makeatletter
\hypersetup{
     	%pagebackref=true,
		pdftitle={\@title}, 
		pdfauthor={\@author},
    	pdfsubject={\imprimirpreambulo},
	    pdfcreator={LaTeX with abnTeX2},
		%pdfkeywords={abnt}{latex}{abntex}{abntex2}{trabalho acadêmico}, 
		colorlinks=true,       		% false: boxed links; true: colored links
    	linkcolor=blue,          	% color of internal links
    	citecolor=blue,        		% color of links to bibliography
    	filecolor=magenta,      	% color of file links
		urlcolor=blue,
		bookmarksdepth=4
}
\makeatother
% -----------------------------------------------------------------------------
