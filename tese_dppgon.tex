\documentclass[
	% -- opções da classe memoir --
    12pt,				% tamanho da fonte (padrao ABNT: 12pt)
	openright,			% capítulos começam em pág ímpar (insere página vazia caso preciso)
	twoside,			% para impressão em verso e anverso. Oposto a oneside
	a4paper,			% tamanho do papel.
	english,			% idioma adicional para hifenização
	brazil,				% o último idioma é o principal do documento
	]{abntex2}
% para mais opções, veja a documentação do projeto abntex2

% Carregar codificação utf-8 para acentuação ANTES DE TODO O RESTO
\usepackage[utf8]{inputenc} % Codificacao do documento (conversão automática dos acentos)
\usepackage{dppgon}  % Carrega o pacote com customizações para o ON
% ------------------------------------------------
% Informações de dados para CAPA e FOLHA DE ROSTO
% ------------------------------------------------
\titulo{Título da Tese}

\autor{Nome do Autor}

\local{Rio de Janeiro}

\data{janeiro de 2015}

\orientador{Dr.~Orientador da Tese}

\coorientador{Dr.~Co-orientador da Tese}

% ---------------------------------------------------
% Informações para compor o preâmbulo da FOLHA DE ROSTO e FOLHA DE APROVAÇÃO
% ---------------------------------------------------
% Tipo do trabalho: Dissertação (mestrado) ou Tese (doutorado)
\tipotrabalho{Dissertação}

% Grau pretendido: Mestre ou Doutor
\graupretendido{Mestre}

% Área de concentração
\area{Astronomia}
% Mestres: Astronomia ou Geofísica
% Doutores: Astrofísica Estelar, Cosmologia, Astronomia Dinâmcia 
% ou Gravimetria, Sismologia, Tectonofísica etc.
%----------------------------------------------------

% ----------------------------------------------
% Informações para compor a FOLHA DE APROVAÇÃO
% ----------------------------------------------
% Data da defesa por extenso
\dataaprovacao{01 de Janeiro de 2015}

% Membros da banca
\membroum{Dr.~Professor Convidado 1}
\instituicaoum{Instituicao do Convidado 1}

\membrodois{Dr.~Professor Convidado 2}
\instituicaodois{Instituição do Convidado 2}

%\membrotres{Dr.~Professor Convidado 3}
%\instituicaotres{Instituição do Convidadeo 3}

%\membroquatro{Dr.~Professor Convidado 4}
%\instituicaoquatro{Instituição do Convidado 4}
% ----------------------------------------------

%-----------------------------------------------
% Informações da clase abntex2 mas não utilizado no modelo Atual
% ----------------------------------------------
% Opção única e carregada automaticamente no preambulo
% -----------
% Atenção para o programa: Astronomia ou Geofísica
\instituicao{%
  Observatório Nacional
  \par
  Divisão de Programas de Pós-Graduação}
% ----------------------------------------------
  % Carrega o arquivo de informações: info.tex

% Carrega arquivo com outras configurações: setup.tex
% -----------------------------------------------------------------------------
% Pacotes fundamentais 
% -----------------------------------------------------------------------------
\usepackage{cmap}				% Mapear caracteres especiais no PDF
\usepackage[T1]{fontenc}		% Selecao de codigos de fonte.
\usepackage{lastpage}			% Usado pela Ficha catalográfica
\usepackage{indentfirst}		% Indenta o primeiro parágrafo de cada seção.
\usepackage{color}				% Controle das cores
\usepackage{graphicx}			% Inclusão de gráficos
\usepackage{hyphenat}           % pode forçar palavras a não serem hyphenizadas
% -----------------------------------------------------------------------------

% -----------------------------------------------------------------------------
% Fonte
% -----------------------------------------------------------------------------
% A fonte padrão do LaTeX é 'Computer Modern' a fonte 'Latin Modern' oferece
% algumas melhorias na renderização de símbolos

\usepackage{lmodern}	        % Usa a fonte Latin Modern
% -----------------------------------------------------------------------------
		
% -----------------------------------------------------------------------------
% Pacotes de citações
% -----------------------------------------------------------------------------
% o pacote abaixo permite citações do tipo: "citado na página X." etc.
\usepackage[brazilian,hyperpageref]{backref} % Paginas com as citações

% veja a documentação do pacote abntex2cite para instruções no padrão de
% citação (veja referência e licensa no arquivo principal desse projeto)
\usepackage[alf]{abntex2cite}	% Citações padrão ABNT do tipo: Autor, Ano
% -----------------------------------------------------------------------------

% -----------------------------------------------------------------------------
% Pacotes adicionais
% -----------------------------------------------------------------------------
\usepackage{lipsum}				% para geração de lero-lero em latim
% -----------------------------------------------------------------------------

% -----------------------------------------------------------------------------
% CONFIGURAÇÕES DE PACOTES
% -----------------------------------------------------------------------------
% Configurações do pacote backref para citações indiretas
% Usado sem a opção hyperpageref de backref
\renewcommand{\backrefpagesname}{Citado na(s) página(s):~}
% Texto padrão antes do número das páginas
\renewcommand{\backref}{}
% Define os textos da citação
\renewcommand*{\backrefalt}[4]{
	\ifcase #1 %
		Nenhuma citação no texto.%
	\or
		Citado na página #2.%
	\else
		Citado #1 vezes nas páginas #2.%
	\fi}%
% -----------------------------------------------------------------------------

% -----------------------------------------------------------------------------
% Espaçamentos entre linhas e parágrafos 
% -----------------------------------------------------------------------------
% O tamanho do parágrafo é dado por:
\setlength{\parindent}{1.3cm}
% Controle do espaçamento entre um parágrafo e outro:
\setlength{\parskip}{0.2cm}  % tente também \onelineskip
% -----------------------------------------------------------------------------

% -----------------------------------------------------------------------------
% informações do PDF
% -----------------------------------------------------------------------------
\makeatletter
\hypersetup{
     	%pagebackref=true,
		pdftitle={\@title}, 
		pdfauthor={\@author},
    	pdfsubject={\imprimirpreambulo},
	    pdfcreator={LaTeX with abnTeX2},
		%pdfkeywords={abnt}{latex}{abntex}{abntex2}{trabalho acadêmico}, 
		colorlinks=true,       		% false: boxed links; true: colored links
    	linkcolor=blue,          	% color of internal links
    	citecolor=blue,        		% color of links to bibliography
    	filecolor=magenta,      	% color of file links
		urlcolor=blue,
		bookmarksdepth=4
}
\makeatother
% -----------------------------------------------------------------------------


% IMPORTANTE: Para impressão mude a cor dos links de AZUL para PRETO
\definecolor{blue}{RGB}{41,5,195} % Links AZUIS
%\definecolor{blue}{RGB}{0,0,0} % Links PRETOS

% ---
% compila o sumário
% ---
\makeindex
% ---

% ----
% Início do documento
% ----
\begin{document}

% Retira espaço extra obsoleto entre as frases.
\frenchspacing

% -----------------------------------------------------------------------------
% ELEMENTOS PRÉ-TEXTUAIS
% -----------------------------------------------------------------------------
% Capa do trabalho
\imprimircapa

% Folha de Rosto (sem estrela indica que não haverá ficha no verso)
\imprimirfolhaderosto
%\imprimirfolhaderosto* (com estrela indica que haverá ficha no verso)

%\imprimirficha  % Isto é somente um exemplo de Ficha Catalográfica
%
% Quando a biblioteca lhe fornecer um PDF com a ficha catalográfica definitiva
% salve-o como no diretório do seu projeto e substitua o comando anterior por:
%
% \begin{fichacatalografica}
%     \includepdf{ficha_catalografica.pdf}
% \end{fichacatalografica}
%

\imprimirfolhadeaprovacao  % Folha de aprovação e espaço para assinaturas

% ---
% Dedicatória
% ---
\begin{dedicatoria}
   \vspace*{\fill}
   \centering
   \noindent
   \textit{ Este trabalho é dedicado às crianças adultas que,\\
   quando pequenas, sonharam em se tornar cientistas.} \vspace*{\fill}
\end{dedicatoria}
% ---
    % opcional
% ---
% Agradecimentos
% ---
\begin{agradecimentos}
Os agradecimentos principais são direcionados à Gerald Weber, Miguel Frasson,
Leslie H. Watter, Bruno Parente Lima, Flávio de Vasconcellos Corrêa, Otavio Real
Salvador, Renato Machnievscz\footnote{Os nomes dos integrantes do primeiro
projeto abn\TeX\ foram extraídos de
\url{http://codigolivre.org.br/projects/abntex/}} e todos aqueles que
contribuíram para que a produção de trabalhos acadêmicos conforme
as normas ABNT com \LaTeX\ fosse possível.

Agradecimentos especiais são direcionados ao Centro de Pesquisa em Arquitetura
da Informação\footnote{\url{http://www.cpai.unb.br/}} da Universidade de
Brasília (CPAI), ao grupo de usuários
\emph{latex-br}\footnote{\url{http://groups.google.com/group/latex-br}} e aos
novos voluntários do grupo
\emph{\abnTeX}\footnote{\url{http://groups.google.com/group/abntex2} e
\url{http://abntex2.googlecode.com/}}~que contribuíram e que ainda
contribuirão para a evolução do \abnTeX.

\end{agradecimentos}
% ---
 % opcional
% ---
% Epígrafe
% ---
\begin{epigrafe}
    \vspace*{\fill}
	\begin{flushright}
		\textit{``Não vos amoldeis às estruturas deste mundo, \\
		mas transformai-vos pela renovação da mente, \\
		a fim de distinguir qual é a vontade de Deus: \\
		o que é bom, o que Lhe é agradável, o que é perfeito.\\
		(Bíblia Sagrada, Romanos 12, 2)}
	\end{flushright}
\end{epigrafe}
% ---
       % opcional

% Resumo em Português
% resumo em português
\setlength{\absparsep}{18pt} % ajusta o espaçamento dos parágrafos do resumo
\begin{resumo}
 Segundo a \citeonline[3.1-3.2]{NBR6028:2003}, o resumo deve ressaltar o
 objetivo, o método, os resultados e as conclusões do documento. A ordem e a extensão
 destes itens dependem do tipo de resumo (informativo ou indicativo) e do
 tratamento que cada item recebe no documento original. O resumo deve ser
 precedido da referência do documento, com exceção do resumo inserido no
 próprio documento. (\ldots) As palavras-chave devem figurar logo abaixo do
 resumo, antecedidas da expressão Palavras-chave:, separadas entre si por
 ponto e finalizadas também por ponto.

 \textbf{Palavras-chaves}: latex. abntex. editoração de texto.

\end{resumo}
         % obrigatório
% Resumo em Inglês
% resumo em inglês
\begin{resumo}[Abstract]
 \begin{otherlanguage*}{english}
   \lipsum[8]

   \vspace{\onelineskip}
 
   \noindent 
   \textbf{Key-words}: latex. abntex. text editoration.
 \end{otherlanguage*}
\end{resumo}
       % obrigatório

% Listas
% ---
% inserir lista de ilustrações
% ---
\pdfbookmark[0]{\listfigurename}{lof}
\listoffigures*
\cleardoublepage
% ---
   % opcional
% ---
% inserir lista de tabelas
% ---
\pdfbookmark[0]{\listtablename}{lot}
\listoftables*
\cleardoublepage
% ---
   % opcional
% ---
% inserir lista de abreviaturas e siglas
% ---
\begin{siglas}
  \item[ABNT] Associação Brasileira de Normas Técnicas
  \item[abnTeX] ABsurdas Normas para TeX
\end{siglas}
% ---
    % opcional
% ---
% inserir lista de símbolos
% ---
\begin{simbolos}
  \item[$ \Gamma $] Letra grega Gama
  \item[$ \Lambda $] Lambda
  \item[$ \zeta $] Letra grega minúscula zeta
  \item[$ \in $] Pertence
\end{simbolos}
% ---
  % opcional

% Sumário (ou Índice)
% ---
% inserir o sumario
% ---
\pdfbookmark[0]{\contentsname}{toc}
\tableofcontents*
\cleardoublepage
% ---
        % obrigatório
% -----------------------------------------------------------------------------

% -----------------------------------------------------------------------------
% ELEMENTOS TEXTUAIS
% -----------------------------------------------------------------------------
\textual % Páginas numeradas daqui em diante

\part{Introdução}
\chapter{Revisão bibliográfica}
\lipsum[1]
\section{Estrelas}
\lipsum[2]
%------------------
% Exemplo de Figura
%------------------
\begin{figure}[htb]
    \caption{\label{fig_circulo}A delimitação do espaço}
    \begin{center}
    \includegraphics[scale=0.45]{figuras/logo_on_positivo-v.pdf}
    \end{center}
    \legend{Fonte: os autores}
\end{figure}

\subsection{Estrelas Massivas}
\lipsum[3]
\chapter{Capitulo 2}
\section{Seção do capitulo 2}
\lipsum[1-3]

\part{Metodologia}
\chapter{Observações}
\lipsum[19]
\section{Redução de Dados}
\lipsum[11]
\subsection{Softwares e Ferramentas}
\lipsum[17]

\part{Resultados}
\chapter{Meus Resultados}
\lipsum[20]
\section{Resultados do Método X}
\lipsum[21]
%-----------------------------------------
% Exemplo de tabela no padrão IBGE-ABNT
%-----------------------------------------
\begin{table}[htb]
    \IBGEtab{%
    \caption{Um Exemplo de tabela alinhada que pode ser longa ou curta,
             conforme padrão IBGE.}%
    \label{tabela-ibge}
    }{%
    \begin{tabular}{ccc}
    \toprule
    Nome & Nascimento & Documento \\
    \midrule \midrule
    Maria da Silva & 11/11/1111 & 111.111.111-11 \\
    \bottomrule
    \end{tabular}%
    }{%
    \fonte{Produzido pelos autores}%
    \nota{Esta é uma nota, que diz que os dados são baseados na
          regressão linear.}%
    \nota[Anotações]{Uma anotação adicional, seguida de várias outras.}%
    }
\end{table}
%------

\subsection{Resultados do Método X e técnica Y}
\lipsum[23]

\part{Conclusão}
\chapter{Discussão dos Resultados}
\lipsum[25]
\section{Repercursões}
\lipsum[8]
\subsection{Sugestão para futuros trabalhos na área}
\lipsum[15]


% -----------------------------------------------------------------------------
% ELEMENTOS PÓS-TEXTUAIS
% -----------------------------------------------------------------------------
\postextual
%% para referências bibliográficas recomenda-se o uso de BibTeX (Google it!)
% nesse caso abaixo as referências devem estar em um arquivo chamado
% "bibliografia.bib" na pasta "pos-texto"
\bibliography{pos-texto/bibliografia}  % obrigatório

% ----------------------------------------------------------
% Apêndices
% ----------------------------------------------------------

% ---
% Inicia os apêndices
% ---
\begin{apendicesenv}

% Imprime uma página indicando o início dos apêndices
\partapendices

% ----------------------------------------------------------
\chapter{Quisque libero justo}
% ----------------------------------------------------------

\lipsum[50]
\end{apendicesenv}
% ---
  % opcional
% ---
% Inicia os anexos
% ---
\begin{anexosenv}

% Imprime uma página indicando o início dos anexos
\partanexos

% ---
\chapter{Morbi ultrices rutrum lorem.}
% ---
\lipsum[30]
\end{anexosenv}
% ---
     % opcional

% -----------------------------------------------------------------------------
% INDICE REMISSIVO (Opcional - Veja documentação da abntex2)
% -----------------------------------------------------------------------------
% \phantompart
% \printindex

\end{document}
