\documentclass[
	% -- opções da classe memoir --
    12pt,				% tamanho da fonte (padrao ABNT: 12pt)
	openright,			% capítulos começam em pág ímpar (insere página vazia caso preciso)
	twoside,			% para impressão em verso e anverso. Oposto a oneside
	a4paper,			% tamanho do papel.
	english,			% idioma adicional para hifenização
	brazil,				% o último idioma é o principal do documento
	]{abntex2}
% para mais opções, veja a documentação do projeto abntex2

% Carrega codificação utf-8 para acentuação - CARREGAR ANTES DE TODO O RESTO
\usepackage[utf8]{inputenc} % Codificacao do documento (conversão automática dos acentos)

% Carrega o pacote com customizações para o ON
\usepackage{dppgon}

% Carrega o arquivo de informações: info.tex
% ---
% Informações de dados para CAPA e FOLHA DE ROSTO
% ---
\titulo{Título da Tese}

\autor{Nome do Autor}

\local{Rio de Janeiro, Brasil}

\data{janeiro de 2015}

\orientador{Prof.~Dr.~Orientador da Tese}

\coorientador{Prof.~Dr.~Co-orientador da Tese}

\instituicao{%
  Observatório Nacional
  \par
  Divisão de Programas de Pós-Graduação
  \par
  Programa de Pós-Graduação em Astronomia}

  % Tipo do trabalho, para ser impresso na ficha catalográfica
% Dissertação (mestrado) ou Tese (doutorado)
\tipotrabalho{Dissertação (mestrado)}

% O preambulo deve conter o tipo do trabalho, o objetivo, 
% o nome da instituição e a área de concentração 
\preambulo{%
  \nohyphens{
  Dissertação apresentada à Divisão de Programas de Pós-Graduação do
  Observatório Nacional, como requisito para a obtenção do
  título de Mestre em Astronomia.
  }
  \mbox{\textbf{Área de Concentração:} Astrofísica Estelar}}
% ---


% Carrega arquivo com outras configurações: setup.tex
\include{setup}

% ---
% compila o sumário
% ---
\makeindex
% ---

% ----
% Início do documento
% ----
\begin{document}

% Retira espaço extra obsoleto entre as frases.
\frenchspacing

% -----------------------------------------------------------------------------
% ELEMENTOS PRÉ-TEXTUAIS
% -----------------------------------------------------------------------------
% Capa do trabalho
\imprimircapa

% Folha de Rosto (a estrela indica que haverá ficha catalográfica no verso)
\imprimirfolhaderosto*

\imprimirficha  % Isto é somente um exemplo de Ficha Catalográfica
% Quando a biblioteca lhe fornecer um PDF com a ficha catalográfica definitiva
% salve-o como no diretório do seu projeto e substitua o comando anterior por:
%
% \begin{fichacatalografica}
%     \includepdf{ficha_catalografica.pdf}
% \end{fichacatalografica}
%

\imprimirfolhadeaprovacao  % Folha de aprovação e espaço para assinaturas

\include{pre-texto/dedicatoria}    % opcional
\include{pre-texto/agradecimentos} % opcional
\include{pre-texto/epigrafe}       % opcional

% Resumo em Português
\include{pre-texto/resumo}         % obrigatório
% Resumo em Inglês
% resumo em inglês
\begin{resumo}[Abstract]
 \begin{otherlanguage*}{english}
   \lipsum[8]

   \vspace{\onelineskip}
 
   \noindent 
   \textbf{Key-words}: latex. abntex. text editoration.
 \end{otherlanguage*}
\end{resumo}
       % obrigatório

% Listas
\include{pre-texto/listafiguras}   % opcional
\include{pre-texto/listatabelas}   % opcional
\include{pre-texto/listasiglas}    % opcional
\include{pre-texto/listasimbolos}  % opcional

% Sumário (ou Índice)
\include{pre-texto/sumario}        % obrigatório
% -----------------------------------------------------------------------------

% -----------------------------------------------------------------------------
% ELEMENTOS TEXTUAIS
% -----------------------------------------------------------------------------
\textual % Páginas numeradas daqui em diante

\part{Introdução}
\chapter{Revisão bibliográfica}
\section{Estrelas}
\subsection{Estrelas Massivas}
  %------------------
% Exemplo de Figura
%------------------
\begin{figure}[htb]
    \caption{\label{fig_circulo}A delimitação do espaço}
    \begin{center}
    \includegraphics[scale=0.75]{figuras/myfig.pdf}
    \end{center}
    \legend{Fonte: os autores}
\end{figure}


\part{Metodologia}
\chapter{Observações}
\lipsum[19]
\section{Redução de Dados}
\lipsum[11]
\subsection{Softwares e Ferramentas}
\lipsum[17]

\part{Resultados}
\chapter{Meus Resultados}
\section{Resultados do Método X}
\subsection{Resultados do Método X e técnica Y}
%-----------------------------------------
% Exemplo de tabela no padrão IBGE-ABNT
%-----------------------------------------
\begin{table}[htb]
    \IBGEtab{%
    \caption{Um Exemplo de tabela alinhada que pode ser longa ou curta,
             conforme padrão IBGE.}%
    \label{tabela-ibge}
    }{%
    \begin{tabular}{ccc}
    \toprule
    Nome & Nascimento & Documento \\
    \midrule \midrule
    Maria da Silva & 11/11/1111 & 111.111.111-11 \\
    \bottomrule
    \end{tabular}%
    }{%
    \fonte{Produzido pelos autores}%
    \nota{Esta é uma nota, que diz que os dados são baseados na
          regressão linear.}%
    \nota[Anotações]{Uma anotação adicional, seguida de várias outras.}%
    }
\end{table}
%------


\part{Conclusão}
\chapter{Discussão dos Resultados}
\lipsum[25]
\section{Repercursões}
\lipsum[8]
\subsection{Sugestão para futuros trabalhos na área}
\lipsum[15]


% -----------------------------------------------------------------------------
% ELEMENTOS PÓS-TEXTUAIS
% -----------------------------------------------------------------------------
\postextual
%% para referências bibliográficas recomenda-se o uso de BibTeX (Google it!)
% nesse caso abaixo as referências devem estar em um arquivo chamado
% "bibliografia.bib" na pasta "pos-texto"
\bibliography{pos-texto/bibliografia}  % obrigatório

% ----------------------------------------------------------
% Apêndices
% ----------------------------------------------------------

% ---
% Inicia os apêndices
% ---
\begin{apendicesenv}

% Imprime uma página indicando o início dos apêndices
\partapendices

% ----------------------------------------------------------
\chapter{Quisque libero justo}
% ----------------------------------------------------------

\lipsum[50]
\end{apendicesenv}
% ---
  % opcional
% ---
% Inicia os anexos
% ---
\begin{anexosenv}

% Imprime uma página indicando o início dos anexos
\partanexos

% ---
\chapter{Morbi ultrices rutrum lorem.}
% ---
\lipsum[30]
\end{anexosenv}
% ---
     % opcional

% -----------------------------------------------------------------------------
% INDICE REMISSIVO (Opcional - Veja documentação da abntex2)
% -----------------------------------------------------------------------------
% \phantompart
% \printindex

\end{document}
